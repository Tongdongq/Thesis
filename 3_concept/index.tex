\documentclass[../main/thesis.tex]{subfiles}

\begin{document}

\chapter{Concept}
\ifdefined\main
\acresetall
MAIN IS TRUE
\newcommand{\codePath}{../3_concept/code/}
\newcommand{\figPath}{../3_concept/figures/}
\else
MAIN IS NOT TRUE

% this file should be included by every subfile, in the \notmain part

% acronyms dont seem to work in subfile mode, make bold to signal
%\newcommand{\ac}[1]{\textbf{#1}}




\begin{acronym}
\acro{NGS}{NGS}{next-generation-sequencing}
\acro{DBG}{DBG}{de-Bruijn graph}
\acro{OLC}{OLC}{overlap-layout-consensus}
\end{acronym}

\newcommand{\code}{code/}

\fi

\section{Pacbio reads}
Daligner finds alignments between long, noisy reads.
Pacific Biosciences has commercially launched its first sequencer in 2011.
It is able to output reads with an average of 1000 bases, which is significantly longer than \ac{NGS} reads \cite{PBlaunch1}.
In 2014, a new polymerase-chemistry combination was released, called P6-C4.
This version can output average read lengths of 10000-15000 bases, and its longest reads can exceed 40000 bases \cite{Longreads}.
While the drawback is that these reads have an error rate of 12-15\%, this can be compensated by the distribution of these errors \cite{Daligner}.
First, the set of reads is a nearly Poisson sampling of the sampled genome.
This implies that there exists a coverage c for every target coverage k, such that every region of the genome is covered k times \cite{Poisson}.
Secondly, the work of Churchill and Waterman \cite{quality} implies that the accuracy of the consensus sequence of k sequences is O($\epsilon^k$), which goes to 0 as k increases.
This means that if the reads are long enough to handle repetitive regions, in principle a near perfect de novo assembly of the genome is possible, given enough coverage \cite{Daligner}.

Important points for de novo DNA sequencing are: what level of coverage is needed for high quality assembly?
And how to build an assembler that is able to deal with high error rates and long reads?
Most previous assemblers work with \ac{NGS} reads, which are much shorter and have much lower error rates.
Some algorithms used in these assemblers, such as \ac{DBG} \cite{DeBruijn} would grow too large for high error rates and long reads.
Since Daligner was build, new methods of using \ac{DBG} with long reads have been developed, but they rely on a short read based \ac{DBG} to correct errors in long reads \cite{DBG1}\cite{DBG2}.

\section{Daligner}

The first step in an \ac{OLC} assembler is usually finding overlaps between reads \cite{OLC}.
BLASR \cite{BLASR} was the only long read aligner at the time, and inpsired Daligner.
It uses the same filtering concept, but with a cache-coherent threaded radix sort to find seeds, instead of a BWT index \cite{BWT}.
\todo{is seed-extend already introduced?}
The most time-consuming step is extending the seed hit to find an alignment.
To do this, Daligner uses a novel method which adaptively computes furthest reaching waves of the older O(nd) algorithm \cite{O_ND}, combined with heuristic trimming and a datastructure that describes a sparse path from the seed hit to the furthest reaching point.

\todo{daligner paper includes a small section on results}
\todo{next part is copied straight from daligner paper}
Daligner performs all-to-all comparison on two input databases $\mathcal{A}$, with $M$ long reads $A_1, A_2,...A_M$ and $\mathcal{B}$, with $N$ long reads $B_1, B_2,...B_N$ over alphabet $\Sigma = 4$
It reports alignments $P = (a,i,g)x(b,j,h)$ such that $len(P) = ((g-i)+(h-j))/2 \ge \tau$ and the optimal alignment between $A_a[i+1,g]$ and $B_b[j+1,h]$ has no more than 2$\epsilon \cdot len(P)$ differences, where a difference can be either an insertion, a deletion or a substitution.
Both $\tau$ and $\epsilon$ are user settable parameters, where $\tau$ is the minimum alignment length and $\epsilon$ the average error rate.
The correlation, or percent identity of the alignment is defined as $1-2\epsilon$.

An edit graph for read $A=a_1a_2...a_m$ and $B=b_1b_2...b_n$ is a graph with $(m+1)(n+1)$ vertices $(i,j) \in [0,M]\times[0,N]$.
It also has three types of edges:
\begin{itemize}
\item deletion edges $(i-1,j) \rightarrow (i,j)$ with label
{\small$\begin{bmatrix}
a_i \\ -
\end{bmatrix}$} if $i > 0$.
\item insertion edges $(i,j-1) \rightarrow (i,j)$ with label 
{\small$\begin{bmatrix}
- \\ b_j
\end{bmatrix}$} if $j > 0$.
\item diagonal edges $(i-1,j-1) \rightarrow (i,j)$ with label
{\small$\begin{bmatrix}
a_i \\ b_j
\end{bmatrix}$} if $i,j > 0$.
\end{itemize}

\todo{insert picture of edit graph}

An alignment between $A[i+1,g]$ and $B[j+1,h]$ is described as a sequence of labels from vertex $(i,j)$ to $(g,h)$.
A diagonal edge can be either be a match edge, when $a_i = b_j$, or a substitution edge.
If a match edge has weight 0, and the other edges have weight 1, the weight of the total path is the number of differences in the alignment it represents.
To find suitable alignments, we have to find a read subset pairs P such that $len(P)\ge \tau$ and the weight of the lowest scoring path between $(i,j)$ and $(g,h)$ in the edit graph of $A_a$ and $B_b$ is not more than $2\epsilon\cdot len(P)$.

\todo{include something about SW and if Daligner is actually SW with certain penalty values?
A naive but exact way to calculate the lowest scoring path is described in the paper from Smith and Waterman \cite{SW}.}

The O(ND) algorithm tries to find progressive waves of furthest reaching (f.r.) points until the endpoint is reached.
The goal is to find longest possible paths starting at a starting point $\rho = (i,j)$ with 0 differences, then 1 difference, then 2 and so on.
After d differences, the possible paths can end in diagonals $\kappa \pm d$, where $\kappa = i-j$ is the diagonal of the starting point.
The furthest reaching point on diagonal $k$ that can be reached from $\rho$ with $d$ differences is called $F_\rho(d,k)$.
A collection of these points for a particular value of $d$ is called the $d$-wave emanating from $\rho$, and defined as $W_\rho(d) = \{F_\rho(d,\kappa-d),...,F_\rho(d,\kappa+d)\}$.
$F_\rho(d,k)$ will be refered to as $F(d,k)$, where $\rho$ is implicitely understood from the context.

In the O(ND) paper it is proven that:
\begin{equation}
F(d,k)=Slide(k, max\{F(d-1,k-1)+(1,0), F(d-1,k)+(1,1), F(d-1,k+1)+(0,1)\}
\end{equation}

where $Slide(k,(i,j)) = (i,j) + max\{\Delta:a_{i+1}a_{i+2}...a_{i+\Delta} = b_{j+1}b_{j+2}...b_{j+\Delta}\}$.

A slide is a path of sequential match edges.
The f.r. $d$-point on diagonal $k$ is calculated by finding the furthest of
\begin{itemize}
\item the f.r. ($d$-1)-point on $k-1$ followed by an insertion
\item the f.r. ($d$-1)-point on $k$ followed by a substitution
\item the f.r. ($d$-1)-point on $k+1$ followed by a deletion
\end{itemize}
and then continuing as far as possible along the slide.
A point $(i,j)$ is furthest when its anti-diagonal $i+j$ is greatest.
The best alignment between reads A and B is the smallest $d$ such that $(m,n)\in W_{(0,0)}(d)$, where $m$ and $n$ are the length of reads A and B.
The O(ND) algorithm cimputes $d$-waves from starting point $(0,0)$ until the end point $(m,n)$ is reached.
The complexity of this algorithm is $O(n+d^2)$ when A and B are non-repetitive sequences \cite{Daligner}.
Because seeds are not always at the beginning, so waves are computed in both forward and reverse direction.
The latter is easily done by reversing the direction of edges in the edit graph.

\subsection{Seeding: concept}
To find suitable starting points for the edit graphs, seeding is done.
A seed is a section where $A[i,g]$ and $B[j,h]$ have a certain high similarity that indicates that these reads probably originate from the same part of the genome.
Finding a seed includes finding matching $k$-mers for every readpair $(a,b)$ with $a\in \mathcal{A}$ and $b\in \mathcal{B}$.
Previous methods to match $k$-mers include Suffix Arrays \cite{SA} and BWT indices \cite{BWT}.

Assuming that the $k$-mer matches are independent, the probablity that a $k$-mer is conserved while sequencing is $\pi = (1-2\epsilon)^k$.
The number of conserved $k$-mers in an alignment of $\tau$ basepairs is a Bernouilli distribution with rate $\pi$, so an average of $\tau\cdot\pi$ $k$-mers are expected in this alignment.
An example: $k=14$, $\epsilon=15\%$ and $\tau = 1500$, then $\pi = .7^{14} = 0.0067$ and the average number of conserved $k$-mers is 10.
Only $.046\%$ of the expected readpairs have 1 or fewer $k$-mers, and only $0.26\%$ have 2 of fewer.
To filter with $99.74\%$ sensitivity, only readpairs with 3 or more $k$-mer matches need to be examined.

The specificity of the filter is increased in two ways:
\begin{itemize}
\item computing the number of $k$-mer matches in diagonal bands of width $2^s$ instead of in the whole reads
\item thresholding on the number of bases in $k$-mer matches, instead of the number of $k$-mers themselves
\end{itemize}

The first way decreases the false positive rate because it only allows readpairs that have their $k$-mer matches relatively close, indicating a smaller region with higher similarity.
The second way relies on the fact that 3 overlapping $k$-mers have a higher probability ($\pi^{1+2/k}$) than 3 disjoint $k$-mers with $3k$ basepairs ($\pi^3$).


The actually find the $k$-mer matches, Daligner uses a sort-merge procedure:
\begin{itemize}
\item Build a list $List_X = \{(kmer(X_x,i),x,i)\}_{x,i}$ of all $k$-mers for database $X \in \{\mathcal{A},\mathcal{B}\}$, where $kmer(R,i)$ is the $k$-mer $R[i-k+1,i]$.
\item Sort both lists in order of their $k$-mers.
\item Merge the two lists and build $List_M=\{(a,b,i,j): kmer(A_a,i) = kmer(B_b,j)\}$ of read and position pairs that have the same $k$-mer.
\item Sort $List_M$ lexicographically on $a$, $b$ and $i$ where $a$ is most significant.
\end{itemize}

All entries for a certain read pair $(a,b)$ are in a continuous segment of the list.
This makes it easy to determine if that read pair has enough $k$-mers and in the right places to constitute a seed hit.
Given parameters $h$ and $s$, each entry $(a,b,i,j)$ for the current read pair is placed in diagonal bands $d = \lfloor(i-j)/2^s\rfloor$ and $d+1$.
Now determine the number of bases in the A-read covered by $k$-mers in each pair of neighbouring diagonal bands.
Note that only bases in matching $k$-mers are counted, not the number of $k$-mers.
When there are k+1 matching consecutive bases, two $k$-mers are generated.
These are less 'valuable' than two non-overlapping $k$-mers.
If $Count(d) \ge h$ for any diagonal band $d$, there is a seed hit for each position $(i,j$) in the band $d$ unless position $i$ was already in the range of a previously calculated local alignment.

\todo{insert image of diagonal bands}

The best values for $h$ and $s$ depend on things like $\epsilon$ and the read lengths.

For Daligner, the default $k$ is 14, and assumed error rate is $0.85$.

\todo{include time analysis?}


\subsection{Seeding: implementation}
Daligner is designed to use multiple threads and use the cache efficiently.
Building and merge the lists in steps 1 and 3 is easy, since only one pass is needed for both actions.
The elements of the lists are compressed into 64-bit integers.
Daligner uses a radix sort \cite{sorting}\cite{radix} to sort the lists in steps 2 and 4.
Each 64-bit integer is a vector of $P=\lceil hbits/B\rceil$, $B$-bit digits ($x_P,x_{P-1},x,...,x_1$) where $B$ is a parameter.
The sort needs $P$ passes, where each pass sorts the elements on a $B$-bit digit $x_i$.
Each pass is done with a bucket sort \cite{sorting} with $2^B$ buckets.
Instead of a linked list, the integers in the list $src$ are moved in precomputed segments $trg[bucket[b]...bucket[b+1]-1]$ of an array $trg[0...N-1]$ with the same size as $src$.
For the $p^{th}$ pass, $bucket[b] = \{i: src[i]_p < b\}$ for each $b \in [0,2^B-1]$.

\lstinputlisting{\codePath radix1.c}

The algorithm takes $O(P(N+2^B))$ time, but $B$ and $P$ are small fixed numbers so it is effectively $O(N)$.
There are a lot of parallel sorting algorithms \cite{parRadix1}\cite{parRadix2}, but Daligner uses a new method that needs half the number of passes that traditional methods use.
Each thread sorts a contiguous segment of size $part = \lceil N/T\rceil$ of $src$ into $trg$, where $T$ is the number of threads.
This means each thread $t \in [0,T-1]$ has a bucket array $bucket[t]$ where $bucket[t][b] = \{i: src[i] < b$ or $src[i] = b$ and $i/part < t\}$.
To reduce the number of passes, a bucket array for the next pass is filled during the current pass.
Each thread counts the number of $B$-bit digits that will be handled in the next pass by itself and every other thread seperately.
If the number at index $i$ will be at index $j$ and in bucket $b$ next pass, then the count in the current pass must not be recorded for the thread $i/part$ that currently sorts the number, but for thread $j/part$ that will sort it in the next pass.
To do this we need to count the number of these events in $next[j/part][i/part][b]$ where $next$ is a $T\times T\times 2^B$ matrix.
If $src[i]$ is about to be moved in the $p^{th}$ pass, then $j = bucket[src[i]_p]$ and $b = src[i]_{p+1}$.
This algorithm takes $O(N/T+T^2)$ time, assuming $B$ and $P$ are fixed.

\lstinputlisting{\codePath radix2.c}

This algorithm is particularly cache efficient because each bucket sort uses two small arrays $bucket$ and $next$ that will usually fit in the L1 cache.
Each bucket sort makes one sweep through $src$ and $2^B$ sweeps through the bucket segments of $trg$.
This totals $2^B+1$ sweeps during each pass.
Each sweep can be prefetched if it is small enough.
This means a smaller $B$ is better for caching behaviour, but this increases the number of passes $hbit/B$ that are needed.
On most processors, e.g. an Intel i7, $B = 8$ gives the fastest radix sort \cite{Daligner}.
The optimal number of threads is more complex, because they usually do not have their own caches.

\todo{insert thread scaling timing measurements from 270MB on hpc1, also describe hpc1 hardware}


\subsection{Local Alignment}
Assuming the filter finds a seed-hit $\rho = (i,j)$, the basic idea is compute furthest reaching waves in forward and reverse direction to find the alignment.
The problem is that as the wave propagates further from $\rho$, it spans wider and wider since it occupies $2d+1$ diagonals.
The final alignment will have only one point from each wave so most of the points are wasted, but we only which ones until the whole alignment is done.
Several strategies are used to trim the width of the wave by removing f.r. points that are unlikely to be in the final alignment.

One of the trimming strategies is stopping when a segment with very low correlation is found.
This referred to as the \textit{regional alignment quality}.
F.r. points with less than $\mathcal{M}$ matches in the last $C$ columns are removed.
For example, when $\epsilon = .15$ then a segment with $M[k] < .55C = 33$ if $C = 60$ is probably not desirable.

To keep track of the matching/mismatching bases, we keep a bitvector $B(d,k)$ that represents the last $C = 60$ columns of the best path from $\rho$ to a given f.r. point $F(d,k)$.
A 0 will denote a mismatch and a 1 a match.
This is easily updated by left-shifting a 0 or 1.
The number of matches $M(d,k)$ can be tracked by observing the bit that is shifted out.
Listing \ref{code:LA} shows pseudo-code that computes $W_\rho(d+1)[low-1,hgh+1]$ from $W_\rho(d)[low,hgh]$ assuming $[low,hgh] \subseteq [\kappa-d,\kappa+d]$ is the trimmed result of the wave $W_\rho(d)$.
Note that the array $W$ only holds the $B$-coordinate $j$ of each f.r. point $(i,j)$, since $i = j + k$.

\lstinputlisting[caption=Local Alignment, label=code:LA]{\codePath la.c}

The second trimming principle involves only keeping f.r. points which are within $\mathcal{L}$ anti-diagonals of the maximal anti-diagonal reached by its wave.
It makes sense that the f.r. point on diagonal $k^*$ that will be in the final alignment is on a greater anti-diagonal $i+k$ than points that are not.
As the other f.r. points in the wave move away from diagonal $k^*$, their anti-diagonal values decrease rapidly, and the wave gets the appearance of an arrowhead.
The higher the correlation of the alignment, the sharper the arrowhead becomes.
This means that points far enough behind the tip can be almost certainly removed.
A value of $\mathcal{L} = 30$ is a universally good value for trimming \cite{Daligner}.
Formally, for each wave computed from the previous trimmed wave, each f.r. point from $[low-1,hgh+1]$ that has either $M[j]<\mathcal{M}$ or $(2W[k^*]+k^*) - (2W[j]+j) > \mathcal{L}$ is removed.
Note that $W[k]$ contains the $B$-coordinate $j$, and $k=i-j$, so $(2W[k]+k) = i+j$.

The Daligner paper does not give a formal proof, but an argument why the average width of the wave $hgh-low$ is constant for any fixed value of $\epsilon$.
This means that the alignment finding algorithm has linear expected time as a function of alignment length.
Consider two f.r. points $A$ and $B$, where $A$ lies on an alignment path with correlation $1-2\epsilon$ or higher, and $B$ does not.
For the next wave, point $A$ moves forward with one difference and then slides on average $\alpha = (1-\epsilon)^2/(1-(1-\epsilon)^2)$ bases.
Point $B$ has no such correlation, so it jumps one difference and then only slides $\beta = 1/(\Sigma-1)$ bases, assuming each base is equally likely.
So an f.r. point $d$ diagonals away from the final path has involved $d$ jumps off the path, and is on average $d(\alpha-\beta)$ anti-diagonals behind the best f.r. point in the wave.
The average width of a wave with an $\mathcal{L}$ lag cutoff is less than $2\mathcal{L}/(\alpha-\beta)$.
This last step is incorrect because the statistics of average random path length under this difference model is complexer than assuming all random steps are the same.
However, there is a definite expected value of path length with $d$ differences, so the basis of the argument holds, although with a different value for $\beta$.
When $\epsilon$ goes to 0, there is a very long slide from the starting point $\rho$.
Each f.r. point not on the path should lag a lot behind the best f.r. point.
As $\alpha$ increases as $\epsilon$ goes to 0, this explains further why the wave becomes very pointy and narrow.

\todo{which last step of the argument is wrong exactly? Probably the formula of \beta}

The alignment finding algorithm ends because either the boundary of the edit graph is reached, or because all f.r. points have failed the regional alignment criterion, this means that the reads are probably not correlated anymore.
In the second case, the best point in the last wave should not be reported as endpoint of the alignment, because the last columns could all be mismatches.
Because the overall path should have an average correlation of $1-2\epsilon$, only a polished point with greatest anti-diagonal can be the end of a path.
A \textit{polished point} is a point for which the last $E\le C$ columns are such that every suffix of the last $E$ columns have a correlation of $1-2\epsilon$ or better, this is called being \textit{suffix positive}.
Daligner keep track of the polished f.r. point with the greatest anti-diagonal during the computation of the waves, it does so by testing if each bit-vector of the leading f.r. points is suffix positive.
Testing bit-vector $e$ can be done in $O(1)$ time by precomputing a table $SP[e]$ with $2^E$ elements.
Define $Score(\emptyset)=0$ and recursively $Score(1b) = Score(b) + \alpha$ and $Score(0b) = Score(b) - \beta$ where $\alpha = 2\epsilon$ and $\beta = 1-2\epsilon$.
Note that if bit-vector $b$ has $m$ matches and $d$ differences, then $Score(b) = \alpha m - \beta d$.
If this is non-negative then $m/(m+d) \ge 1-2\epsilon$, which means $b$ has a correlation of $1-2\epsilon$ or higher.
Now let $SP[e] = min\{Score(b): b$ is a suffix of $e\}$.
The table $SP$ can be built in linear time by computing $Score$ over the trie \cite{trie1}\cite{trie2} of all $E$-bit vectors and taking the minimum of each path of the trie.

However for large values of $E$, say 30, the table $SP$ gets too big.
To solve this, a size $D$, say 15, is chosen for which the $SP$ table is reasonable.
Consider an $E$-bit vector $e$ that consists of $X=E/D$, $D$-bit segments $e_X\cdot e_{X-1}\cdot ... \cdot e_1$.
Precompute table $SP$ as before, but now only for $D$ bits, and a table $SC$ for $D$-bit vectors as well, where $SC[b] = Score(b)$.
With these two $2^D$ tables it takes $O(X)$ time to determine if the longer bit-vector $e$ is suffix positive by calculating if $Polish(X)$ is true with the following recurrences:
\begin{align}
Score(x) = &\begin{cases}
Score(x-1) + SC[e_x] &\text{ if } x \ge 1 \\
0&\text{ if } x = 0
\end{cases} \\
Polish(x) = &\begin{cases}
Polish(x-1) \text{ and } Score(x-1) + SP[e_x] \ge 0 &\text{ if } x \ge 1 \\
true &\text{ if } x = 0
\end{cases}
\end{align}

\todo{include summary of Daligner algorithm?}
\todo{include summary of Daligner results wrt sensitivity and running time?}




\section{Darwin}
Darwin is a hardware-accelerated framework for genomic sequence alignment \cite{Darwin1}\cite{Darwin2}.
It consists of a filter called D-SOFT (Diagonal-band Seed Overlapping based Filtration Technique), which finds seeds, and GACT (Genome Alignment using Constant memory Traceback), which performs alignment between reads of arbitrary length, using constant traceback.
The fact that GACT uses constant traceback makes it very suitable for hardware acceleration.
Darwin boasts a 39.000x more energy-efficient approach then software, and a 15.000x speedup for reference-guided alignment for third generation reads using an ASIC.
On an FPGA, Darwin has a 183.0x speedup for reference-guided and 19.9x speedup for denovo over state-of-the-art software.


\subsection{D-SOFT}
Like Daligner, D-SOFT filters by counting bases in Kmer matches.
For each band of diagonals (bin), a particular threshold must be exceeded to constitute a seed hit, thus invoking the GACT algorithm.
When using D-SOFT to do de-novo alignment, each read is padded with unknown nucleotides (N), so that each read starts at the start of a bin.
Figure \ref{fig:D_SOFT} illustrates the working of D-SOFT for k = 4, N = 10, h = 8, N$_{B}$ = 6.
N is the number of Kmer matches that are considered, h is the minimum number of unique bases that the Kmer matches in a particular bin must have, N$_{B}$ is the number of bins.
The start positions of the matches are marked by red dots, the rest of the seed is a red line.
Bin 1 contains six consecutive matching bases, which form three Kmer matches.
Bin 3 also contains three Kmer matches, but they contain nine matching bases, for h = 8, only bin 3 produces a seed.
As oppose to Daligner, D-SOFT only considers the Kmer matches from one bin, not taking its direct neighbours into account.

\figC{width=.7\textwidth}{D_SOFT.png}{Illustration of D-SOFT algorithm, source: \cite{Darwin2}}{fig:D_SOFT}

First, a lookup table for the Kmer all Kmers and their positions is made.
This can be done with any implementation, like seed position tables, suffix arrays or BWT or FM based indices.
\todo{add citations for those implementations?}

D-SOFT uses two arrays, \textit{bp\_count} and \textit{last\_hit\_pos}.
Both have N$_{B}$ values, and store the number of bases in Kmer matches, and the last seed hit position for that bin, respectively.
\todo{explain D-SOFT algorithm? Maybe include code?}

\subsection{GACT}
Normal Smith-Waterman will take too much memory and time for long reads, because the complexity is quadratric for the length of the reads.
Numerous efforts to accelerate Smith-Waterman have been made, both by using hardware and by using heuristics.
\todo{name accelerated variants here, or in Chapter 2 Background?}
\todo{banded alignment is mentioned later this subsection}
Even when using heuristics, the memory requirements are still linear.
GACT finds the alignment between reads of arbitrary length using constant traceback memory.
It performs normal Smith-Waterman on a submatrix called a tile of size T, and then moves to the next tile, which overlaps with the previous tile with O bases.
For reasonable values for T and O, GACT produces an optimal result.

The algorithm for left extension is shown in Listing \ref{code:GACT}.
Positions $i\_curr$ and $j\_curr$ are produced by D-SOFT.
The start and end position of the current tile are stored in $i,j\_start$ and $i,j\_curr$ respectively.
The traceback path of the whole left extension is kept in $tb_left$.
The function $Align()$ uses modified Smith-Waterman to compute the optimal alignment between subsequences $R\_tile$ and $Q\_tile$, with traceback starting from the bottom-right cell, except for the first tile, where traceback starts from the highest-scoring cell.
$Align()$ returns the tile's alignment score, the number of bases in R and Q aligned by this tile ($i,j\_off$), the traceback pointers $tb$ and the position of the highest-scoring cell (ignored except for the first cell).
$Align()$ also limits $i,j\_off$ to at most T - O bases, to ensure the next tile overlaps by at least O bases.
The left extension finishes when it hits the end of $R$ or $Q$, or when traceback cannot add any bases to the existing alignment.
The memory needed for the traceback is $O(T^2)$, which is constant since T is chosen up front.
The traceback for the whole alignment $tb\_left$ is linear with read length, but not performance critical.
GACT calculates the full alignment strings and uses these to calculate the score of the whole alignment.
The right extension operates on the reverse of $R$ and $Q$.

\lstinputlisting[caption=GACT, label=code:GACT]{\codePath GACT.c}

Figure \ref{fig:GACT} shows a left extension from GACT, using T = 4 and O = 1.
For the first tile (T1), traceback starts at the highest-scoring cell (green), the other tiles start at the bottom-right cell (yellow), where the traceback from the previous tile ends.
The traceback is ended earlier for higher overlaps, so that the tiles will overlap.


\figC{width=.7\textwidth}{GACT.png}{Illustration of GACT algorithm, source: \cite{Darwin2}}{fig:GACT}


The performance of GACT is linear ($O(max(m,n)\cdot T)$) with respect to the length of the reads.
It is more suited to long reads than banded alignment, because banded alignment uses a static band around the main diagonal.
GACT allows for flexible bands, since the position of the new tile depends on the traceback path, this is useful for long reads that have high indel rates.
\todo{flexible bands are better when indels are present, should I prove/cite that it does not matter when only having substitutions?}




\end{document}















