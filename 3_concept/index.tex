
\documentclass[../thesis.tex]{../ce}

\begin{document}

\chapter{Concept}
\ifdefined\main
\acresetall
MAIN IS TRUE
\else
MAIN IS NOT TRUE
\fi

\section{Pacbio reads}
Daligner finds alignments between long, noisy reads.
Pacific Biosciences has commercially launched its first sequencer in 2011.
It is able to output reads with an average of 1000 bases, which is significantly longer than \ac{NGS} reads  \cite{PBlaunch1}.
In 2014, a new polymerase-chemistry combination was released, called P6-C4.
This version can output average read lengths of 10000-15000 bases, and its longest reads can exceed 40000 bases \cite{Longreads}.
While the drawback is that these reads have an error rate of 12-15\%, this can be compensated by the distribution of these errors \cite{Daligner}.
First, the set of reads is a nearly Poisson sampling of the sampled genome.
This implies that there exists a coverage c for every target coverage k, such that every region of the genome is covered k times \cite{Poisson}.
Secondly, the work of Churchill and Waterman \cite{quality} implies that the accuracy of the consensus sequence of k sequences is O($\epsilon^k$), which goes to 0 as k increases.
This means that if the reads are long enough to handle repetitive regions, in principle a near perfect de novo assembly of the genome is possible, given enough coverage.

Important points for de novo DNA sequencing are: what level of coverage is needed for high quality assembly?
And how to build an assembler that is able to deal with high error rates and long reads?
Most previous assemblers work with \ac{NGS} reads, which are much shorter and have much lower error rates.
Some algorithms used in these assemblers, such as \ac{DBG} \cite{DeBruijn} would grow too large for high error rates and long reads.
Since Daligner was build, new methods of using \ac{DBG} with long reads have been developed, but they rely on a short read based \ac{DBG} to correct errors in long reads \cite{DBG1}\cite{DBG2}.

\section{Daligner}

The first step in an \ac{OLC} assembler is usually finding overlaps between reads \cite{OLC}.
BLASR \cite{BLASR} was the only long read aligner at the time, and inpsired Daligner.
It uses the same filtering concept, but with a cache-coherent threaded radix sort to find seeds, instead of a BWT index \cite{BWT}.
%TODO is seed-extend already introduced?
The most time-consuming step is extending the seed hit to find an alignment.
To do this, Daligner uses a novel method which adaptively computes furthest reaching waves of the older O(nd) algorithm \cite{O_ND}, combined with heuristic trimming and a datastructure that describes a sparse path from the seed hit to the furthest reaching point.

%TODO daligner paper includes a small section on results
%TODO next part is copied straight from daligner paper
Daligner performs all-to-all comparison on two input databases $A$, with $M$ long reads $A_1, A_2,...A_M$ and $B$, with $N$ long reads $B_1, B_2,...B_N$ over alphabet $\Sigma = 4$
 It reports alignments $P = (a,i,g)x(b,j,h)$ such that $len(P) = ((g-i)+(h-j))/2 \ge \tau$ and the optimal alignment between $A_a[i+1,g]$ and $B_b[j+1,h]$ has no more than 2$\epsilon \cdot len(P)$ differences, where a difference can be either an insertion, a deletion or a substitution.
Both $\tau$ and $\epsilon$ are user settable parameters, where $\tau$ is the minimum alignment length and $\epsilon$ the average error rate.
The correlation, or percent identity of the alignment is defined as $1-2\epsilon$.





\end{document}















