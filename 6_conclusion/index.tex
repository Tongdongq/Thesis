\documentclass[../main/thesis.tex]{subfiles}


\begin{document}

\chapter{Conclusion}
\ifdefined\main
\else

% this file should be included by every subfile, in the \notmain part

% acronyms dont seem to work in subfile mode, make bold to signal
%\newcommand{\ac}[1]{\textbf{#1}}




\begin{acronym}
\acro{NGS}{NGS}{next-generation-sequencing}
\acro{DBG}{DBG}{de-Bruijn graph}
\end{acronym}

\newcommand{\code}{code/}

\fi

\section{Performance}
GPU implementations for both Daligner and Darwin were made.
For Daligner, the CPU version is clearly the better choice.
Considering the nature of this algorithm, this is not very surprising.
Darwin showed better results on GPU, with substantial speedups. %TODO 'up to xxx'
However, Daligner can produce more sensitive alignments in a shorter time.
The main reason is the amount of work that needs to be done by the algorithms, Daligner calculates around 26 diagonals on average, Darwin calculates $T^2=102400$ SW elements.
The main advantage of Darwin is that is calculates an actual SW score, instead of doing match/mismatches.
This means Darwin can be configured to prefer substitutions or gaps, and customize it to suit a particular sequencing technology.

A clear tradeoff between sensitivity and specificity can be observed from the experiments.
This shows that both algorithms can be configured to suit the input requirements of later stages in the assembly pipeline.


\section{Future Work}

%TODO future work
% - Yatish told me about transitive overlaps:
% - - if A and B overlap, and B and C too, then A and C probably also overlap

% stuff here depends on the profile results to identify bottlenecks



A way to interpret the produced overlaps more efficiently is to use transitive overlaps.
If reads A and B overlap, and B and C too, then A and C could also overlap.
However, this should only result in marginal improvements of the sensitivity, since the overlap between A and C should already be found.
\todo{Yatish told me this, how to cite?}



The CPU version could be optimized as well.
The current multi-threaded version has no load balancing.
All CPU threads can a roughly equal amount of reads that must be test by D-SOFT.
The number of seeds that emerge can vary (observed values are 17000 and 30655), this is partially caused by the different read lengths.
Another reason for the imbalance is that the extension phase for each seed can have a different duration.
One way to solve this is to create a large list with all the seeds, possibly partitioned so that the threads can write to this list in parallel.
When starting the GACT phase, each thread gets a batch of reads to extend.
When a thread is done with its batch, or has less seeds than GPU threads, it gets another batch from the large list.

Since GASAL is designed to work with affine gap penalties, it uses four arrays to keep track of the score.
If a linear gap penalty is used, it can be reduced to two.
The calculation of the maximum score can also be simplified.
Since this the main part of the algorithm, this could reduce the runtime significantly.



\end{document}









