\documentclass[../main/thesis.tex]{subfiles}

\begin{document}

\chapter{Background}
\ifdefined\main
\acresetall
MAIN IS TRUE
\newcommand{\codePath}{../2_background/code/}
\newcommand{\figPath}{../2_background/figures/}
\else
MAIN IS NOT TRUE

% this file should be included by every subfile, in the \notmain part

% acronyms dont seem to work in subfile mode, make bold to signal
%\newcommand{\ac}[1]{\textbf{#1}}




\begin{acronym}
\acro{NGS}{NGS}{next-generation-sequencing}
\acro{DBG}{DBG}{de-Bruijn graph}
\end{acronym}

\newcommand{\code}{code/}

\fi

\section{DNA sequencing}


%TODO also talk about OpenCL?
%TODO why GPU in the first place? What do they look like, what are they good at?
%TODO CUDA was released in 2006, source: programming guide


\section{GPU processing}
A GPU is a Graphics Processing Unit, it is a processor that is mainly used to perform video processing.
This type of processing often includes rotation and translation of objects in a space, calculating shadows and rendering images to display on a monitor.
They contain many cores that allow it to perform parallelizable tasks very quickly.
A GPGPU, or General Purpose GPU can be programmed to perform tasks that are different from video processing.
An algorithm like matrix addition is easily parallelized by assigning a matrix cell to each thread.
Each thread can perform the addition in parallel, let this take one cycle.
A sequential implementation would have taken $N\times M$ cycles, where $N$ and $M$ are the dimensions of the matrices.

CPUs usually have large caches and a complex instruction set and execution that includes out-of-order execution and branch prediction \cite.

GPUs cannot operate on their own, they must be guided by a CPU.
A general workflow using a GPU is shown in Figure \ref{gpu:workflow}.

%TODO include gpu:workflow

\begin{itemize}
\item Copy data from CPU to GPU
\item Let GPU process the data
\item Copy results from GPU to CPU
\end{itemize}

The functions that run on a GPU are called \textit{kernels}, and are usually launched by a CPU.
Kernels can also be called from other kernels.


%TODO maybe divide into subsections
\section{CUDA}
CUDA is a parallel computing platform that allows people to use Nvidia GPUs for their own applications.
Developers can write functions that will execute on the GPU called \textit{kernels}, these are launched from a CPU function.
The GPU is referred to as \textit{device} and the CPU as \textit{host}.
Kernels for CUDA are written in C++.

Kernels can be launched from the CPU in a \textit{grid} with a certain number of thread blocks or \textit{blocks} and \textit{threads}.
A grid is one-/two- or three-dimensional, and has an array of blocks in each direction.
Each block itself is also one-/two- or three-dimensional, and has an array of threads in each direction.
Figure \ref{cuda:hierarchy} shows a full grid.

%TODO picture of hierarchy \label{cuda:hierarchy}
% possible source: http://international.download.nvidia.com/pdf/kepler/NVIDIA-Kepler-GK110-GK210-Architecture-Whitepaper.pdf

Each thread executes the kernel code, although they usually operate on different data.
Threads in a block can communicate via shared memory.

On a hardware level, an NVIDIA GPU is divided into Streaming Multiprocessors (SMX).
Each SMX contains a number of cores, or Streaming Processors (SP), these are the basic building blocks and perform the actual calculations.
Each block is assigned to at most one SMX.
This block's threads are then executed as warps, with 32 threads per warp.
Each SMX has multiple warp schedulers, so multiple warps can run in parallel on an SMX.
All threads in a warp must execute the same instruction, if a thread is the only to take a branch, the other threads must wait until the branch is completed, this is called divergence.
Memory operations are also executed in parallel, this means that all threads try to read/write to the memory in parallel.
If the addresses are next to eachother, only one memory transaction is needed, since a transaction processes a whole memory line.
This is known as coalescing.

%TODO include figure of coalescing


%TODO include figure of hardware
%https://www.techpowerup.com/img/12-05-17/155a.jpg

%Tesla-Fermi-Kepler-Maxwell-Pascal-Volta


\subsection{Memory hierarchy}
GPUs have several different memory types and levels.
Not all of these are accessible from the host or from other components of the memory hierarchy.
\begin{itemize}
\item Global memory: The largest and slowest memory, located outside the chip. Can be read and written from the host. Best used for coalesced operations.
\item Local memory: Each thread has private memory for when registers are not enough. It is stored in global memory, so accesses are very slow.
\item Registers: The fastest memory available, located on-chip. A thread has a maximum number of registers available depending on Compute Capability.
\item Shared memory: Each SMX has shared memory, it can be accesses by every thread in every block on the SMX. This can be used by threads in a block to communicate. It is located on-chip, so very fast.
\item Constant memory: The host can initialize this memory, the kernel can only read it. Reading is as fast as reading a register, but only when all addresses of a half-warp are the same, otherwise reading is serialized. It resides off-chip, but is cached on-chip.
\item Texture memory: Can only be written from the host. Resides off-chip, but is cached on-chip, like constant memory. The main feature about the texture memory is that it is cached for 2D spatial locality. Figure \ref{fig:texture_cache} shows an access pattern that would not be cached with a typical scheme.
\item L2 cache: Behaves as cache for device memory and is shared among all SMXs. The cache line size is 32.
\item L1 cache: 
\end{itemize}

%TODO include figure of memory hierarchy

%\includegraphics[scale=.2]{\fig texture_cache.png}
\figC{scale=.2}{texture_cache.png}{This access pattern has spatial locality, and would be cached in the texture cache.}{fig:texture_cache}

\subsection{Streams}
When copying data to or from the GPU, there are usually no kernels running in a naive implementation.
This means that if the copy time is large, the overall efficiency is quite low.
These operations can be overlapped or pipelined by using streams.
%TODO include picture of streams/pipelining





%TODO include this section? this work mainly focuses on GPU computation, instead of heterogeneous computation
\section{Heterogeneous computing}
%\cite{hetero_survey}




\end{document}










