\documentclass[../main/thesis.tex]{subfiles}

\begin{document}

\chapter{Background}
\ifdefined\main
\acresetall
MAIN IS TRUE
\newcommand{\code}{../2_background/code/}
\else
MAIN IS NOT TRUE

% this file should be included by every subfile, in the \notmain part

% acronyms dont seem to work in subfile mode, make bold to signal
%\newcommand{\ac}[1]{\textbf{#1}}




\begin{acronym}
\acro{NGS}{NGS}{next-generation-sequencing}
\acro{DBG}{DBG}{de-Bruijn graph}
\end{acronym}

\newcommand{\code}{code/}

\fi

\section{DNA sequencing}


%TODO also talk about OpenCL?
%TODO why GPU in the first place? What do they look like, what are they good at?
%TODO CUDA was released in 2006, source: programming guide


\section{GPU processing}
A GPU is a Graphics Processing Unit, it is a processor that is mainly used to perform video processing.
This type of processing often includes rotation and translation of objects in a space, calculating shadows and rendering images to display on a monitor.
They contain many cores that allow it to perform parallelizable tasks very quickly.
A GPGPU, or General Purpose GPU can be programmed to perform tasks that are different from video processing.
An algorithm like matrix addition is easily parallelized by assigning a matrix cell to each thread.
Each thread can perform the addition in parallel, let this take one cycle.
A sequential implementation would have taken $N\times M$ cycles, where $N$ and $M$ are the dimensions of the matrices.

CPUs usually have large caches and a complex instruction set and execution that includes out-of-order execution and branch prediction \cite.

GPUs cannot operate on their own, they must be guided by a CPU.
A general workflow using a GPU is shown in Figure \ref{gpu:workflow}.

%TODO include gpu:workflow

\begin{itemize}
\item Copy data from CPU to GPU
\item Let GPU process the data
\item Copy results from GPU to CPU
\end{itemize}

The functions that run on a GPU are called \textit{kernels}, and are usually launched by a CPU.
Kernels can also be called from other kernels.



\section{CUDA}
CUDA is a parallel computing platform that allows people to use Nvidia GPUs for their own applications.
Developers can write functions that will execute on the GPU called \textit{kernels}, these are launched from a CPU function.
The GPU is referred to as \textit{device} and the CPU as \textit{host}.
Kernels for CUDA are written in C++.

Kernels can be launched from the CPU in a \textit{grid} with a certain number of thread blocks or \textit{blocks} and \textit{threads}.
A grid is one-/two- or three-dimensional, and has an array of blocks in each direction.
Each block itself is also one-/two- or three-dimensional, and has an array of threads in each direction.
Figure \ref{cuda:hierarchy} shows a full grid.

%TODO picture of hierarchy \label{cuda:hierarchy}

Each thread executes the kernel, although they usually operate on different data.













\end{document}










