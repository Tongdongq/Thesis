\documentclass[../main/thesis.tex]{subfiles}

\begin{document}

\chapter{Specification}
\ifdefined\main
\acresetall
MAIN IS TRUE
\newcommand{\code}{../3_concept/code/}
\else
MAIN IS NOT TRUE

% this file should be included by every subfile, in the \notmain part

% acronyms dont seem to work in subfile mode, make bold to signal
%\newcommand{\ac}[1]{\textbf{#1}}




\begin{acronym}
\acro{NGS}{NGS}{next-generation-sequencing}
\acro{DBG}{DBG}{de-Bruijn graph}
\acro{OLC}{OLC}{overlap-layout-consensus}
\end{acronym}

\newcommand{\code}{code/}

\fi

\section{Profiling}

MCprof \cite{mcprof1}\cite{mcprof2} was used to profile Daligner, the results are shown in Figures \ref{prof1} and \ref{prof2}.
The algorithm consists of three main phases: seeding, filtering and aligning.
The seeding phase takes part in the lex\_sort(), count\_thread() and merge\_thread() functions.
The filter implementation is not very complex, and is combined with the alignment part in report\_thread(), this function starts the actual alignment in the form of Local\_Alignment() when a suitable seed is found.
Each seed is aligned in forward and reverse direction.


%TODO insert pictures of profiling
% \label{prof1}, \label{prof2}


Since Local\_Alignment() is the most timeconsuming function, that will be the first one the be considered for acceleration.
Each Local\_Alignment() call originates from a separate readpair $A, B$, so executing multiple Local\_Alignment() instances at the same time should not be a problem.
This is why Daligner uses multiple threads, each one running the filter-align function report\_thread() on its own list of readpairs.

In order to launch multiple Local\_Alignment() functions from one kernel, they have to be prepared.


\end{document}





